\documentclass{article}
\usepackage[margin=1in]{geometry}
\usepackage{ctex}
\usepackage{amsmath}
\usepackage{bm}
\usepackage{enumerate}
\usepackage{enumitem}
\usepackage{array}
\usepackage{caption}
\newcolumntype{P}[1]{>{\centering\arraybackslash}p{#1}}
\renewcommand {\thetable} {6.\arabic{table}}

\begin{document}
\captionsetup{labelformat=default,labelsep=space}
\begin{center}\Huge\bfseries
    HE
\end{center}
\begin{enumerate}[label=(\arabic*)]
    \item 质量方程(Rayleigh equation)
    \begin{equation}
        \label{eq6.1}
        \tag{6.1}
        D-U_0 = V_0 \sqrt{\frac{p-p_0}{V_0-V}}
    \end{equation}
    \item 动量方程
    \begin{equation}
        \label{eq6.2}
        \tag{6.2}
        U-U_0=(V_0-V)\sqrt{\frac{p-p_0}{V_0-V}}
    \end{equation}
    \item 能量方程(Rankine-Hugoniot equation)
    \begin{equation}
        \label{eq6.3}
        \tag{6.3}
        e-e_0 = \frac{1}{2} (p+p_0) (V_0-V)
    \end{equation}
    \item 用压力和比容表示的爆炸气体状态方程\par
    \begin{center}
        C-J假设(Chapman-Jouguet Hypothesis)
    \end{center}
    对于稳定传播的爆炸波,Rayleigh曲线必须与气态反应生成物的Hugoniot曲线相于C-J点。
    (Mader, 1979; 1998) \par
    对C-J点处的压力(C-J压力)而言,HE的初始压力$P_0$非常小,而且可以被忽略。
    (Zhang, 1976; Mader, 1998)
\end{enumerate}

控制方程
\begin{equation}
    \label{eq6.4}
    \tag{6.4}
    \left\{
    \begin{aligned}
        \frac{\texttt{d} \rho}{\texttt{d} t} &= -\rho \bm{\nabla} \cdot \bm{v} \\
        \frac{\texttt{d} \bm{v}}{\texttt{d} t} &= -\frac{1}{\rho} \bm{\nabla} \cdot p \\
        \frac{\texttt{d} e}{\texttt{d} t} &= -\frac{p}{\rho} \bm{\nabla} \cdot \bm{v} \\
        p &= p(\rho, e)
    \end{aligned}
    \right.
\end{equation}

状态方程(State equation)

TNT爆轰速度为$6930m/s$
对于爆炸气体,可使用标准的Jones-Wilkins-Lee状态方程(Dobratz, 1981)。
爆炸气体的压力可由下式得出
\begin{equation}
    \label{eq6.5}
    \tag{6.5}
    p = A \left(1-\frac{\omega \eta}{R_1}\right) e^{-\frac{R_1}{\eta}}
      + B \left(1-\frac{\omega \eta}{R_2}\right) e^{-\frac{R_2}{\eta}}
      + \omega \eta \rho_0 e
\end{equation}
\begin{table}[htbp]
    \centering
    \caption{TNT材料参数和由实验所获得的状态方程中的相应系数}\label{tab:table1}
    \begin{tabular}{P{2cm} P{5cm} P{5cm}}
        \hline
        符号 & 意义 & 值 \\ \hline
        $\rho_0$ & 初始密度 & 1630$kg/m^3$ \\
        D & 爆轰速度 & 6930$m/s$ \\
        $P_{CJ}$ & C-J压力 & $2.1\times10^10Pa$ \\
        A & 拟合系数 & $3.712\times10^11Pa$ \\
        A & 拟合系数 & $3.712\times10^11Pa$ \\
        $R_1$ & 拟合系数 & 4.15 \\
        $R_2$ & 拟合系数 & 0.95 \\
        $\omega$ & 拟合系数 & 0.30 \\
        $E_0$ & 单位质量的爆轰能量 & $4.29\times10^6J/kg$ \\ \hline
    \end{tabular}
\end{table}
\end{document}